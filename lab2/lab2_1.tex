% Options for packages loaded elsewhere
\PassOptionsToPackage{unicode}{hyperref}
\PassOptionsToPackage{hyphens}{url}
%
\documentclass[
]{article}
\usepackage{amsmath,amssymb}
\usepackage{iftex}
\ifPDFTeX
  \usepackage[T1]{fontenc}
  \usepackage[utf8]{inputenc}
  \usepackage{textcomp} % provide euro and other symbols
\else % if luatex or xetex
  \usepackage{unicode-math} % this also loads fontspec
  \defaultfontfeatures{Scale=MatchLowercase}
  \defaultfontfeatures[\rmfamily]{Ligatures=TeX,Scale=1}
\fi
\usepackage{lmodern}
\ifPDFTeX\else
  % xetex/luatex font selection
\fi
% Use upquote if available, for straight quotes in verbatim environments
\IfFileExists{upquote.sty}{\usepackage{upquote}}{}
\IfFileExists{microtype.sty}{% use microtype if available
  \usepackage[]{microtype}
  \UseMicrotypeSet[protrusion]{basicmath} % disable protrusion for tt fonts
}{}
\makeatletter
\@ifundefined{KOMAClassName}{% if non-KOMA class
  \IfFileExists{parskip.sty}{%
    \usepackage{parskip}
  }{% else
    \setlength{\parindent}{0pt}
    \setlength{\parskip}{6pt plus 2pt minus 1pt}}
}{% if KOMA class
  \KOMAoptions{parskip=half}}
\makeatother
\usepackage{xcolor}
\usepackage[margin=1in]{geometry}
\usepackage{color}
\usepackage{fancyvrb}
\newcommand{\VerbBar}{|}
\newcommand{\VERB}{\Verb[commandchars=\\\{\}]}
\DefineVerbatimEnvironment{Highlighting}{Verbatim}{commandchars=\\\{\}}
% Add ',fontsize=\small' for more characters per line
\usepackage{framed}
\definecolor{shadecolor}{RGB}{248,248,248}
\newenvironment{Shaded}{\begin{snugshade}}{\end{snugshade}}
\newcommand{\AlertTok}[1]{\textcolor[rgb]{0.94,0.16,0.16}{#1}}
\newcommand{\AnnotationTok}[1]{\textcolor[rgb]{0.56,0.35,0.01}{\textbf{\textit{#1}}}}
\newcommand{\AttributeTok}[1]{\textcolor[rgb]{0.13,0.29,0.53}{#1}}
\newcommand{\BaseNTok}[1]{\textcolor[rgb]{0.00,0.00,0.81}{#1}}
\newcommand{\BuiltInTok}[1]{#1}
\newcommand{\CharTok}[1]{\textcolor[rgb]{0.31,0.60,0.02}{#1}}
\newcommand{\CommentTok}[1]{\textcolor[rgb]{0.56,0.35,0.01}{\textit{#1}}}
\newcommand{\CommentVarTok}[1]{\textcolor[rgb]{0.56,0.35,0.01}{\textbf{\textit{#1}}}}
\newcommand{\ConstantTok}[1]{\textcolor[rgb]{0.56,0.35,0.01}{#1}}
\newcommand{\ControlFlowTok}[1]{\textcolor[rgb]{0.13,0.29,0.53}{\textbf{#1}}}
\newcommand{\DataTypeTok}[1]{\textcolor[rgb]{0.13,0.29,0.53}{#1}}
\newcommand{\DecValTok}[1]{\textcolor[rgb]{0.00,0.00,0.81}{#1}}
\newcommand{\DocumentationTok}[1]{\textcolor[rgb]{0.56,0.35,0.01}{\textbf{\textit{#1}}}}
\newcommand{\ErrorTok}[1]{\textcolor[rgb]{0.64,0.00,0.00}{\textbf{#1}}}
\newcommand{\ExtensionTok}[1]{#1}
\newcommand{\FloatTok}[1]{\textcolor[rgb]{0.00,0.00,0.81}{#1}}
\newcommand{\FunctionTok}[1]{\textcolor[rgb]{0.13,0.29,0.53}{\textbf{#1}}}
\newcommand{\ImportTok}[1]{#1}
\newcommand{\InformationTok}[1]{\textcolor[rgb]{0.56,0.35,0.01}{\textbf{\textit{#1}}}}
\newcommand{\KeywordTok}[1]{\textcolor[rgb]{0.13,0.29,0.53}{\textbf{#1}}}
\newcommand{\NormalTok}[1]{#1}
\newcommand{\OperatorTok}[1]{\textcolor[rgb]{0.81,0.36,0.00}{\textbf{#1}}}
\newcommand{\OtherTok}[1]{\textcolor[rgb]{0.56,0.35,0.01}{#1}}
\newcommand{\PreprocessorTok}[1]{\textcolor[rgb]{0.56,0.35,0.01}{\textit{#1}}}
\newcommand{\RegionMarkerTok}[1]{#1}
\newcommand{\SpecialCharTok}[1]{\textcolor[rgb]{0.81,0.36,0.00}{\textbf{#1}}}
\newcommand{\SpecialStringTok}[1]{\textcolor[rgb]{0.31,0.60,0.02}{#1}}
\newcommand{\StringTok}[1]{\textcolor[rgb]{0.31,0.60,0.02}{#1}}
\newcommand{\VariableTok}[1]{\textcolor[rgb]{0.00,0.00,0.00}{#1}}
\newcommand{\VerbatimStringTok}[1]{\textcolor[rgb]{0.31,0.60,0.02}{#1}}
\newcommand{\WarningTok}[1]{\textcolor[rgb]{0.56,0.35,0.01}{\textbf{\textit{#1}}}}
\usepackage{graphicx}
\makeatletter
\def\maxwidth{\ifdim\Gin@nat@width>\linewidth\linewidth\else\Gin@nat@width\fi}
\def\maxheight{\ifdim\Gin@nat@height>\textheight\textheight\else\Gin@nat@height\fi}
\makeatother
% Scale images if necessary, so that they will not overflow the page
% margins by default, and it is still possible to overwrite the defaults
% using explicit options in \includegraphics[width, height, ...]{}
\setkeys{Gin}{width=\maxwidth,height=\maxheight,keepaspectratio}
% Set default figure placement to htbp
\makeatletter
\def\fps@figure{htbp}
\makeatother
\setlength{\emergencystretch}{3em} % prevent overfull lines
\providecommand{\tightlist}{%
  \setlength{\itemsep}{0pt}\setlength{\parskip}{0pt}}
\setcounter{secnumdepth}{-\maxdimen} % remove section numbering
\ifLuaTeX
  \usepackage{selnolig}  % disable illegal ligatures
\fi
\IfFileExists{bookmark.sty}{\usepackage{bookmark}}{\usepackage{hyperref}}
\IfFileExists{xurl.sty}{\usepackage{xurl}}{} % add URL line breaks if available
\urlstyle{same}
\hypersetup{
  pdftitle={Objects, Classes \& NAs},
  hidelinks,
  pdfcreator={LaTeX via pandoc}}

\title{Objects, Classes \& NAs}
\author{}
\date{\vspace{-2.5em}2024-01-16}

\begin{document}
\maketitle

{
\setcounter{tocdepth}{2}
\tableofcontents
}
\hypertarget{learning-goals}{%
\subsection{Learning Goals}\label{learning-goals}}

\emph{At the end of this exercise, you will be able to:}\\
1. Define an object in R.\\
2. Use objects to perform calculations.\\
3. Explain the difference between data classes in R.\\
4. Use R to identify the class of specific data.\\
5. Define NA in R.\\
6. Determine whether or not data have NA values.

\hypertarget{objects}{%
\subsection{Objects}\label{objects}}

In order to access the potential of R we need to assign values or other
types of data to \texttt{objects}. There is a specific format that I
want you to follow, so please pay close attention.

Assign a value to object `x'. The `\textless-' symbol is read as `gets'.
In this case, x gets 42. Make sure that you are in the environment panel
and you should see the value associated with `x'. On a mac, you can push
\texttt{option} and \texttt{-} to automatically generate the gets
symbol.

\begin{Shaded}
\begin{Highlighting}[]
\NormalTok{x }\OtherTok{\textless{}{-}} \DecValTok{42}
\end{Highlighting}
\end{Shaded}

To print the object to the screen, just type x.

\begin{Shaded}
\begin{Highlighting}[]
\NormalTok{x}
\end{Highlighting}
\end{Shaded}

\begin{verbatim}
## [1] 42
\end{verbatim}

\begin{Shaded}
\begin{Highlighting}[]
\NormalTok{y}\OtherTok{\textless{}{-}} \DecValTok{30} 
\end{Highlighting}
\end{Shaded}

\begin{Shaded}
\begin{Highlighting}[]
\NormalTok{z}\OtherTok{=}\DecValTok{30} \CommentTok{\#do not use equal sign for object assignments }
\NormalTok{q }\OtherTok{\textless{}{-}} \DecValTok{42}
\end{Highlighting}
\end{Shaded}

Once an object has been created, you can do things with them.

\begin{Shaded}
\begin{Highlighting}[]
\NormalTok{treatment }\OtherTok{\textless{}{-}} \DecValTok{36}
\NormalTok{control }\OtherTok{\textless{}{-}} \DecValTok{38}
\end{Highlighting}
\end{Shaded}

Here we make a new object \texttt{my\_experiment} that is the sum of the
treatment and control. Notice that I use \texttt{\_} and not spaces.

\begin{Shaded}
\begin{Highlighting}[]
\NormalTok{my\_experiment }\OtherTok{\textless{}{-}} \FunctionTok{sum}\NormalTok{(treatment, control)}
\NormalTok{my\_experiment}
\end{Highlighting}
\end{Shaded}

\begin{verbatim}
## [1] 74
\end{verbatim}

\begin{Shaded}
\begin{Highlighting}[]
\NormalTok{treatment}\SpecialCharTok{+}\NormalTok{control}
\end{Highlighting}
\end{Shaded}

\begin{verbatim}
## [1] 74
\end{verbatim}

\hypertarget{nomenclature}{%
\subsection{Nomenclature}\label{nomenclature}}

We need to be careful about nomenclature when we write code. R allows us
to give almost any name we want to an object, but there are exceptions.
For example, we don't want to give a name to an object that is the same
as a function in R.

\begin{Shaded}
\begin{Highlighting}[]
\ControlFlowTok{else} \OtherTok{\textless{}{-}} \DecValTok{12}
\end{Highlighting}
\end{Shaded}

We get an error here because \texttt{else} is a function in R. You also
don't want to give names that might get confused with functions;
i.e.~you can assign a value to `mean' but this could become confusing
because mean is used as a function.

\begin{Shaded}
\begin{Highlighting}[]
\DecValTok{2}\SpecialCharTok{+}\DecValTok{8}\SpecialCharTok{+}\DecValTok{2}
\end{Highlighting}
\end{Shaded}

\begin{verbatim}
## [1] 12
\end{verbatim}

\begin{Shaded}
\begin{Highlighting}[]
\NormalTok{(}\DecValTok{2}\SpecialCharTok{+}\DecValTok{8}\SpecialCharTok{+}\DecValTok{2}\NormalTok{)}\SpecialCharTok{/}\DecValTok{3}
\end{Highlighting}
\end{Shaded}

\begin{verbatim}
## [1] 4
\end{verbatim}

\begin{Shaded}
\begin{Highlighting}[]
\FunctionTok{mean}\NormalTok{(}\DecValTok{2}\SpecialCharTok{+}\DecValTok{8}\SpecialCharTok{+}\DecValTok{2}\NormalTok{)}
\end{Highlighting}
\end{Shaded}

\begin{verbatim}
## [1] 12
\end{verbatim}

\hypertarget{practice}{%
\subsection{Practice}\label{practice}}

\begin{enumerate}
\def\labelenumi{\arabic{enumi}.}
\tightlist
\item
  Create three new objects, \texttt{venom\_GT}, \texttt{chiron}, and
  \texttt{veyron}. These are the fastest cars in the world. Assign each
  car to its top speed. The venom\_GT can go 270, chiron is 261, and
  veyron is 268.
\end{enumerate}

\begin{Shaded}
\begin{Highlighting}[]
\NormalTok{mean }\OtherTok{\textless{}{-}} \DecValTok{20} \CommentTok{\#Test, do not use!}
\end{Highlighting}
\end{Shaded}

\begin{Shaded}
\begin{Highlighting}[]
\NormalTok{venom\_GT }\OtherTok{\textless{}{-}} \DecValTok{270}
\NormalTok{chiron }\OtherTok{\textless{}{-}} \DecValTok{261}
\NormalTok{veyron }\OtherTok{\textless{}{-}} \DecValTok{268} 
\end{Highlighting}
\end{Shaded}

\begin{enumerate}
\def\labelenumi{\arabic{enumi}.}
\setcounter{enumi}{1}
\tightlist
\item
  Use arithmetic to calculate the mean top speed for the cars.
\end{enumerate}

\begin{Shaded}
\begin{Highlighting}[]
\NormalTok{(venom\_GT}\SpecialCharTok{+}\NormalTok{chiron}\SpecialCharTok{+}\NormalTok{veyron)}\SpecialCharTok{/}\DecValTok{3}
\end{Highlighting}
\end{Shaded}

\begin{verbatim}
## [1] 266.3333
\end{verbatim}

\begin{enumerate}
\def\labelenumi{\arabic{enumi}.}
\setcounter{enumi}{2}
\tightlist
\item
  Use the function \texttt{mean()} to calculate the mean top speed for
  the cars.
\end{enumerate}

\begin{Shaded}
\begin{Highlighting}[]
\FunctionTok{mean}\NormalTok{(venom\_GT}\SpecialCharTok{+}\NormalTok{chiron}\SpecialCharTok{+}\NormalTok{veyron)}
\end{Highlighting}
\end{Shaded}

\begin{verbatim}
## [1] 799
\end{verbatim}

\begin{Shaded}
\begin{Highlighting}[]
\NormalTok{cars }\OtherTok{\textless{}{-}} \FunctionTok{c}\NormalTok{(venom\_GT, chiron, veyron)}
\FunctionTok{mean}\NormalTok{ (cars)}
\end{Highlighting}
\end{Shaded}

\begin{verbatim}
## [1] 266.3333
\end{verbatim}

\begin{Shaded}
\begin{Highlighting}[]
\FunctionTok{mean}\NormalTok{(}\FunctionTok{c}\NormalTok{(venom\_GT, chiron, veyron))}
\end{Highlighting}
\end{Shaded}

\begin{verbatim}
## [1] 266.3333
\end{verbatim}

\hypertarget{types-of-data}{%
\subsection{Types of Data}\label{types-of-data}}

There are five frequently used \texttt{classes} of data: 1. numeric, 2.
integer, 3. character, 4. logical, 5. complex.

\begin{Shaded}
\begin{Highlighting}[]
\NormalTok{my\_numeric }\OtherTok{\textless{}{-}} \DecValTok{42}
\NormalTok{my\_integer }\OtherTok{\textless{}{-}}\NormalTok{ 2L }\CommentTok{\#adding an L automatically denotes an integer}
\NormalTok{my\_character }\OtherTok{\textless{}{-}} \StringTok{"universe"}
\NormalTok{my\_logical }\OtherTok{\textless{}{-}} \ConstantTok{FALSE}
\NormalTok{my\_complex }\OtherTok{\textless{}{-}} \DecValTok{2}\SpecialCharTok{+}\NormalTok{4i}
\end{Highlighting}
\end{Shaded}

To find out what type of data you are working with, use the
\texttt{class()} function. This is important because sometimes we will
need to change the type of data to perform certain analyses.

\begin{Shaded}
\begin{Highlighting}[]
\FunctionTok{class}\NormalTok{(my\_numeric)}
\end{Highlighting}
\end{Shaded}

\begin{verbatim}
## [1] "numeric"
\end{verbatim}

\begin{Shaded}
\begin{Highlighting}[]
\FunctionTok{class}\NormalTok{ (my\_integer)}
\end{Highlighting}
\end{Shaded}

\begin{verbatim}
## [1] "integer"
\end{verbatim}

You can use the \texttt{is()} and \texttt{as()} functions to clarify or
specify a type of data.

\begin{Shaded}
\begin{Highlighting}[]
\FunctionTok{is.integer}\NormalTok{(my\_numeric) }\CommentTok{\#is my\_numeric (the object) an integer?}
\end{Highlighting}
\end{Shaded}

\begin{verbatim}
## [1] FALSE
\end{verbatim}

\begin{Shaded}
\begin{Highlighting}[]
\NormalTok{my\_integer }\OtherTok{\textless{}{-}} 
  \FunctionTok{as.integer}\NormalTok{(my\_numeric) }\CommentTok{\#create a new object specified as an integer}
\end{Highlighting}
\end{Shaded}

\begin{Shaded}
\begin{Highlighting}[]
\FunctionTok{is.integer}\NormalTok{(my\_integer) }\CommentTok{\#is my\_numeric an integer?}
\end{Highlighting}
\end{Shaded}

\begin{verbatim}
## [1] TRUE
\end{verbatim}

\hypertarget{missing-data}{%
\subsection{Missing Data}\label{missing-data}}

R has a special way to designate missing data, the NA. NA values in R
have specific properties which are very useful if your data contains any
missing values. Later this quarter we will have a session focused on
dealing with NAs.

NA values are used to designate missing data. \texttt{is.na} or
\texttt{anyNA} are useful functions when dealing with NAs in data.

\begin{Shaded}
\begin{Highlighting}[]
\NormalTok{my\_missing }\OtherTok{\textless{}{-}} \ConstantTok{NA}
\end{Highlighting}
\end{Shaded}

\begin{Shaded}
\begin{Highlighting}[]
\FunctionTok{is.na}\NormalTok{(my\_missing)}
\end{Highlighting}
\end{Shaded}

\begin{verbatim}
## [1] TRUE
\end{verbatim}

\begin{Shaded}
\begin{Highlighting}[]
\FunctionTok{anyNA}\NormalTok{(my\_missing)}
\end{Highlighting}
\end{Shaded}

\begin{verbatim}
## [1] TRUE
\end{verbatim}

\hypertarget{practice-1}{%
\subsection{Practice}\label{practice-1}}

\begin{enumerate}
\def\labelenumi{\arabic{enumi}.}
\tightlist
\item
  Let's create a vector that includes some missing data (we will discuss
  vectors more in part 2). For now, run the following code chunk.
\end{enumerate}

\begin{Shaded}
\begin{Highlighting}[]
\NormalTok{new\_vector }\OtherTok{\textless{}{-}} \FunctionTok{c}\NormalTok{(}\DecValTok{7}\NormalTok{, }\FloatTok{6.2}\NormalTok{, }\DecValTok{5}\NormalTok{, }\DecValTok{9}\NormalTok{, }\ConstantTok{NA}\NormalTok{, }\DecValTok{4}\NormalTok{, }\FloatTok{9.8}\NormalTok{, }\DecValTok{7}\NormalTok{, }\DecValTok{3}\NormalTok{, }\DecValTok{2}\NormalTok{)}
\end{Highlighting}
\end{Shaded}

\begin{enumerate}
\def\labelenumi{\arabic{enumi}.}
\setcounter{enumi}{1}
\tightlist
\item
  As we did in homework 1, calculate the mean of \texttt{new\_vector}.
\end{enumerate}

\begin{Shaded}
\begin{Highlighting}[]
\FunctionTok{mean}\NormalTok{( new\_vector)}
\end{Highlighting}
\end{Shaded}

\begin{verbatim}
## [1] NA
\end{verbatim}

\begin{enumerate}
\def\labelenumi{\arabic{enumi}.}
\setcounter{enumi}{2}
\tightlist
\item
  How do you interpret this result? What does this mean about NAs?
\end{enumerate}

RStudio is unable to calculate the mean of the data set if there is
missing data.

\begin{enumerate}
\def\labelenumi{\arabic{enumi}.}
\setcounter{enumi}{3}
\tightlist
\item
  Recalculate the mean using the following code chunk. Why is the
  useful?
\end{enumerate}

\begin{Shaded}
\begin{Highlighting}[]
\FunctionTok{mean}\NormalTok{(new\_vector, }\AttributeTok{na.rm=}\NormalTok{T) }\CommentTok{\#na.rm removes the NA values in the vector}
\end{Highlighting}
\end{Shaded}

\begin{verbatim}
## [1] 5.888889
\end{verbatim}

\hypertarget{thats-it-lets-take-a-break-and-then-move-on-to-part-2}{%
\subsection{That's it! Let's take a break and then move on to part
2!}\label{thats-it-lets-take-a-break-and-then-move-on-to-part-2}}

--\textgreater{}\href{https://jmledford3115.github.io/datascibiol/}{Home}

\end{document}
